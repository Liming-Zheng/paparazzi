\section{Matrix $3\times3$ / Rotation Matrix}
The indices of a matrix are zero-indexed, i.e. the first element in the matrix has the row and column number \textbf{Zero}.
\mynote{I choosed to start the indices with 1.}
\subsection{Definition}
The matrix is represented as an array with the length 9.
\begin{equation}
\mat M = \begin{pmatrix}
m_0 & m_1 & m_2 \\
m_3 & m_4 & m_5 \\
m_6 & m_7 & m_8
\end{pmatrix} = \begin{pmatrix}
m[0] & m[1] & m[2] \\
m[3] & m[4] & m[5] \\
m[6] & m[7] & m[8]
\end{pmatrix}
\end{equation}
It is available for the following simple types:\\
\begin{tabular}{c|c|c}
type		& struct Mat	& struct RMat	\\ \hline
int32\_t	& Int32Mat33	& Int32RMat		\\
float		& FloatMat33	& FloatRMat		\\
double		& DoubleMat33	& DoubleRMat
\end{tabular}



\subsection{= Assigning}
\subsubsection*{$\mat M = \mat 0$}
\begin{equation}
\mat M = \begin{pmatrix}
0 & 0 & 0 \\
0 & 0 & 0 \\
0 & 0 & 0
\end{pmatrix}
\end{equation}
\inHfile{FLOAT\_MAT33\_ZERO(m)}{pprz\_algebra\_float}
\inHfile{FLOAT\_RMAT\_ZERO(m)}{pprz\_algebra\_float}

\subsubsection*{$a_{ij}$ elements}
Accessing an element is able with
\inHfile{MAT33\_ELMT(m, row, col)}{pprz\_algebra}
\inHfile{RMAT\_ELMT(m, row, col)}{pprz\_algebra}

\subsubsection*{$\mat M = diag(d_{00}, d_{11}, d_{22})$}
\begin{equation}
\mat M = \begin{pmatrix}
d_{00} & 0 & 0 \\
0 & d_{11} & 0 \\
0 & 0 & d_{22}
\end{pmatrix}
\end{equation}
\inHfile{FLOAT\_MAT33\_DIAG(m, d00, d11, d22)}{pprz\_algebra\_float}

\subsubsection*{$\mat{A} = \mat{B}$}
\begin{equation}
\mat {mat1} = \mat {mat2}
\end{equation}
\inHfile{MAT33\_COPY(mat1, mat2)}{pprz\_algebra}
\inHfile{RMAT\_COPY(o, i)}{pprz\_algebra}


\subsubsection*{$\mat M_{b2a} = \inv{\mat M_{a2b}} = \transp{\mat M_{a2b}}$}
\begin{equation}
\mat M_{b2a} = \inv{\mat M_{a2b}} = \transp{\mat M_{a2b}}
\end{equation}
\inHfile{FLOAT\_RMAT\_INV(m\_b2a, m\_a2b)}{pprz\_algebra\_float}

\subsection{- Subtraction}
\subsubsection*{$\mat C = \mat A - \mat B$}
\begin{equation}
\mat C = \mat A - \mat B
\end{equation}
\inHfile{RMAT\_DIFF(c, a, b)}{pprz\_algebra}
For bigger matrices you have to spezify the number of rows (\texttt{i}) and the number of columns (\texttt{j}).
\inHfile{MAT\_SUB(i, j, C, A, B)}{pprz\_simple\_matrix}



\subsection{$\multiplication$ Multiplication}
\subsubsection*{$\mat M_{a2c} = \mat M_{b2c} \multiplication \mat M_{a2b}$ with a Matrix (composition)}
Makes a matrix-multiplication with additional Right-Shift about the decimal point.
\mynote{Not quite sure about that}
\begin{equation}
\mat M_{a2c} = \mat M_{b2c} \multiplication \mat M_{a2b}
\end{equation}
\inHfile{INT32\_RMAT\_COMP(m\_a2c, m\_a2b, m\_b2c)}{pprz\_algebra\_int}
\inHfile{FLOAT\_RMAT\_COMP(m\_a2c, m\_a2b, m\_b2c)}{pprz\_algebra\_float}
and with the inverse matrix
\begin{equation}
\mat M_{a2b} = \inv{\mat M_{b2c}} \multiplication \mat M_{a2c}
\end{equation}
\inHfile{INT32\_RMAT\_COMP\_INV(m\_a2b, m\_a2c, m\_b2c)}{pprz\_algebra\_int}
\inHfile{FLOAT\_RMAT\_COMP\_INV(m\_a2b, m\_a2c, m\_b2c)}{pprz\_algebra\_float}
Multiplication is also possible with bigger matrices
\begin{equation}
\mat C_{i \cross j} = \mat A_{i \cross k} \multiplication \transp{\mat B_{j \cross k}}
\end{equation}
\inHfile{MAT\_MUL\_T(i, k, j, C, A, B)}{pprz\_simple\_matrix}
or
\begin{equation}
\mat C_{i \cross j} = \mat A_{i \cross k} \multiplication \mat B_{k \cross j}
\end{equation}
\inHfile{MAT\_MUL(i, k, j, C, A, B)}{pprz\_simple\_matrix}


\subsection{Transformation from a Matrix}
\subsubsection*{to euler angles}
\mynote{This is only for the 321-convention}
The rotation matrix from euler angles is known
\begin{equation}
\mat R_m = \begin{pmatrix}
cos(\Pitch)cos(\Yaw)									& cos(\Pitch)sin(\Yaw)									& -sin(\Pitch)			\\
sin(\Roll)sin(\Pitch)cos(\Yaw) - cos(\Roll)cos(\Yaw)	& sin(\Roll)sin(\Pitch)sin(\Yaw) + cos(\Roll)cos(\Yaw)	& sin(\Roll)cos(\Pitch)	\\
cos(\Roll)sin(\Pitch)cos(\Yaw) + sin(\Roll)sin(\Yaw)	& cos(\Roll)sin(\Pitch)sin(\Yaw) - sin(\Roll)cos(\Yaw)	& cos(\Roll)cos(\Pitch)
\end{pmatrix}
\end{equation}
and the extraction is done vice versa.
\begin{equation}
\eu e = \begin{pmatrix}\Roll \\ \Pitch \\ \Yaw \end{pmatrix} = 
\begin{pmatrix}
\arctan2(r_{23}, r_{33}) \\
-\arcsin(r_{13}) \\
\arctan2(r_{12}, r_{11})
\end{pmatrix}
\end{equation}
\inHfile{INT32\_EULERS\_OF\_RMAT(e, rm)}{pprz\_algebra\_int}


\subsubsection*{to a quaternion}
Since the construction of a matrix from a quaternion is known
\begin{equation}
\mat R_m = \begin{pmatrix}
1-2(q_y^2 + q_z^2)		& 2(q_xq_y-q_iq_z)		& 2(q_xq_z + q_iq_y) \\
2(q_xq_y + q_iq_z)		& 1-2(q_x^2 + q_z^2)	& 2(q_yq_z - q_iq_x) \\
2(q_xq_z - q_iq_y)		& 2(q_yq_z+q_iq_x)		& 1-2(q_x^2 + q_y^2)	
\end{pmatrix},
\end{equation}
the extraction of a quaternion is done vice versa. But there are obviously many opportunities to extract the quaternion. They differ in the way which element of the quaternion is extracted from the diaognal elements $r_{11}$, $r_{22}$ and $r_{33}$  of the matrix.
\begin{equation}
1 = q_i^2+q_x^2+q_y^2+q_z^2
\end{equation}
\textbf{First case}
\begin{eqnarray}
\zeta = \sqrt{1 + (r_{11}+r_{22}+r_{33})}  =  \sqrt{1 + (3q_i^2 -q_x^2 -q_y^2 -q_z^2)} = \sqrt{4 q_i^2} \\
q_i = \tfrac 1 2 \zeta \\
q_x =  \tfrac 1 {2 \zeta} (r_{23}-r_{32}) \\
q_y =  \tfrac 1 {2 \zeta} (r_{31}-r_{13}) \\
q_z =  \tfrac 1 {2 \zeta} (r_{12}-r_{21})
\end{eqnarray}
\textbf{Second case}
\begin{eqnarray}
\zeta = \sqrt{1 + (r_{11}-r_{22}-r_{33})}  =  \sqrt{1 + (-q_i^2+3q_x^2 -q_y^2 -q_z^2)} = \sqrt{4 q_x^2} \\
q_i =  \tfrac 1 {2 \zeta} (r_{23}-r_{32}) \\
q_x = \tfrac 1 2 \zeta \\
q_y =  \tfrac 1 {2 \zeta} (r_{12}+r_{21}) \\
q_z =  \tfrac 1 {2 \zeta} (r_{31}+r_{13})
\end{eqnarray}
\textbf{Third case}
\begin{eqnarray}
\zeta = \sqrt{1 + (-r_{11}+r_{22}-r_{33})}  =  \sqrt{1 + (-q_i^2 -q_x^2+3q_y^2 -q_z^2)} = \sqrt{4 q_y^2} \\
q_i =  \tfrac 1 {2 \zeta} (r_{31}-r_{13}) \\
q_x =  \tfrac 1 {2 \zeta} (r_{12}+r_{21}) \\
q_y = \tfrac 1 2 \zeta \\
q_z =  \tfrac 1 {2 \zeta} (r_{23}+r_{32})
\end{eqnarray}
\textbf{Fourth case}
\begin{eqnarray}
\zeta = \sqrt{1 + (-r_{11}-r_{22}+r_{33})}  =  \sqrt{1 + (-q_i^2 -q_x^2 -q_y^2+3q_z^2)} = \sqrt{4 q_z^2} \\
q_i =  \tfrac 1 {2 \zeta} (r_{12}-r_{21}) \\
q_x =  \tfrac 1 {2 \zeta} (r_{31}+r_{13}) \\
q_y =  \tfrac 1 {2 \zeta} (r_{23}+r_{32}) \\
q_z = \tfrac 1 2 \zeta
\end{eqnarray}
All are mathematicaly equivalent but numerically different. To avoid complex numbers and singularities the case with the biggest $\zeta$ should be choosen. 
\inHfile{INT32\_QUAT\_OF\_RMAT(q, r)}{pprz\_algebra\_int}
\inHfile{FLOAT\_QUAT\_OF\_RMAT(q, r)}{pprz\_algebra\_float}



\subsection{Tranformation to a Matrix}
\subsubsection*{from an axis and an angle}
With a known axis of rotation $\vect u$ and an angle $\alpha$ it is possible to compute a rotational matrix with
\begin{equation}
\mat R_m = \begin{pmatrix}
0 & -u_z & u_y \\
u_z & 0 & -u_x \\
-u_y & u_x & 0
\end{pmatrix} \sin \alpha + \left( \eye - \vect u \transp{\vect u} \right)\cos \alpha +  \vect u \transp{\vect u}.
\end{equation}
Please note again, that all angles are from the perspective of the aircraft (see section \ref{Important definition}). Therefore the angle is defined ngeative, leading to
\begin{equation}
\mat R_m = \begin{pmatrix}
0 & u_z & -u_y \\
-u_z & 0 & u_x \\
u_y & -u_x & 0
\end{pmatrix} \sin \alpha + \left( \eye - \vect u \transp{\vect u} \right)\cos \alpha +  \vect u \transp{\vect u}.
\end{equation}
Rearranging this equation leads to
\begin{equation}
\mat R_m = \begin{pmatrix}
u_x^2+(1-u_x^2)\cos \alpha				& u_xu_y(1-\cos \alpha) + u_z \sin \alpha	& u_xu_z(1-\cos \alpha) - u_y \sin \alpha \\
u_xu_y(1-\cos \alpha) - u_z \sin \alpha	& u_y^2+(1-u_y^2)\cos \alpha				& u_yu_z(1-\cos \alpha) + u_x \sin \alpha \\
u_xu_z(1-\cos \alpha) + u_y \sin \alpha & u_yu_z(1-\cos \alpha) - u_x \sin \alpha	& u_z^2+(1-u_z^2)\cos \alpha
\end{pmatrix}.
\end{equation}
\inHfile{FLOAT\_RMAT\_OF\_AXIS\_ANGLE(rm, uv, an)}{pprz\_algebra\_float}

\subsubsection*{from euler angles}
The transformation from euler angles $ \eu e$ to a rotational matrix depends on the order of rotation. Here, the default order is 321, which means first \Yawc{Yaw} (about the \emph{third} axis), then \Pitchc{Pitch} (the \emph{second} axis) and finally \Rollc{Roll}(the \emph{first} axis). Please note the important definition about perspectives (page \ref{Important definition}).
\begin{equation}
\mat R_m = \begin{pmatrix}
cos(\Pitch)cos(\Yaw)									& cos(\Pitch)sin(\Yaw)									& -sin(\Pitch)			\\
sin(\Roll)sin(\Pitch)cos(\Yaw) - cos(\Roll)cos(\Yaw)	& sin(\Roll)sin(\Pitch)sin(\Yaw) + cos(\Roll)cos(\Yaw)	& sin(\Roll)cos(\Pitch)	\\
cos(\Roll)sin(\Pitch)cos(\Yaw) + sin(\Roll)sin(\Yaw)	& cos(\Roll)sin(\Pitch)sin(\Yaw) - sin(\Roll)cos(\Yaw)	& cos(\Roll)cos(\Pitch)
\end{pmatrix}\end{equation}

\inHfile{INT32\_RMAT\_OF\_EULERS(rm, e)}{pprz\_algebra\_int}
\inHfile{INT32\_RMAT\_OF\_EULERS\_321(rm, e)}{pprz\_algebra\_int}
\inHfile{FLOAT\_RMAT\_OF\_EULERS(rm, e)}{pprz\_algebra\_float}
\inHfile{FLOAT\_RMAT\_OF\_EULERS\_321(rm, e)}{pprz\_algebra\_float}
You can also choose the 312 definition (First \Yawc{Yaw}, then  \Rollc{Roll} then \Pitchc{Pitch} $\Rightarrow \mat R(\Yaw) \mat R(\Roll)  \mat R(\Pitch)$). Again, remember the different order and sign:
\begin{equation}
\mat R_m = \mat R(-\Pitch) \mat R(-\Roll)  \mat R(-\Yaw)
\end{equation}
\begin{equation}
\mat R_m = \begin{pmatrix}
cos(\Pitch)cos(\Yaw)-sin(\Roll)sin(\Pitch)sin(\Yaw)		& cos(\Pitch)sin(\Yaw) + sin(\Roll)sin(\Pitch)cos(\Yaw)	& -cos(\Roll)sin(\Pitch) \\
-cos(\Roll)sin(\Yaw)									& cos(\Roll)cos(\Yaw)									& sin(\Roll)		\\
sin(\Pitch)cos(\Yaw) + sin(\Roll)cos(\Pitch)sin(\Yaw)	& sin(\Pitch)sin(\Yaw)-sin(\Roll)cos(\Pitch)cos(\Yaw)	& cos(\Roll)cos(\Pitch)
\end{pmatrix}\end{equation}
\inHfile{INT32\_RMAT\_OF\_EULERS\_312(rm, e)}{pprz\_algebra\_int}
\inHfile{FLOAT\_RMAT\_OF\_EULERS\_312(rm, e)}{pprz\_algebra\_float}
\inHfile{DOUBLE\_RMAT\_OF\_EULERS\_312(rm, e)}{pprz\_algebra\_float}


\subsubsection*{from a quaternion}
The most common definition for this transformation is
\begin{equation}
\mat R_m = \begin{pmatrix}
1-2(q_y^2 + q_z^2)		& 2(q_xq_y-q_iq_z)		& 2(q_xq_z + q_iq_y) \\
2(q_xq_y + q_iq_z)		& 1-2(q_x^2 + q_z^2)	& 2(q_yq_z - q_iq_x) \\
2(q_xq_z - q_iq_y)		& 2(q_yq_z+q_iq_x)		& 1-2(q_x^2 + q_y^2)	
\end{pmatrix}.
\end{equation}

\inHfile{INT32\_RMAT\_OF\_QUAT(rm, q)}{pprz\_algebra\_int}
\inHfile{FLOAT\_RMAT\_OF\_QUAT(rm, q)}{pprz\_algebra\_float}
\mynote{I called the quicker function "INT32\_RMAT\_OF\_QUAT\_QUICKER"}





\subsection{Other}
\subsubsection*{Trace}
\begin{equation}
tr(\mat{R}_m) = a_{11} + a_{22} + a_{33}
\end{equation}
\inHfile{RMAT\_TRACE(rm)}{pprz\_algebra}

\subsubsection*{$\norm{\norm{\mat M}}_F$ Norm (Frobenius)}
Calculates the Frobenius Norm of a matrix
\begin{equation}
\norm{\norm{\mat M}}_F = \sqrt{\sum_{i=1}^3 \sum_{i=1}^3 m_{ij}^2 }
\end{equation}
\inHfile{FLOAT\_RMAT\_NORM(m)}{pprz\_algebra\_float}

\subsection*{$\inv{\mat A} $ Inversion}
The inversion of a 3-by-3 matrix is made using the adjugate matrix and the determinant:
\begin{equation}
\inv{\mat A} = \frac{adj(\mat A)}{det(\mat A} = \frac{1}{det{\mat A}} \begin{pmatrix}
a_{22}a_{33}-a_{23}a_{32}&a_{13}a_{32}-a_{12}a_{33}&a_{12}a_{23}-a_{13}a_{22}\\
a_{23}a_{31}-a_{21}a_{33}&a_{11}a_{33}-a_{13}a_{31}&a_{13}a_{21}-a_{11}a_{23}\\
a_{21}a_{31}-a_{22}a_{31}&a_{12}a_{31}-a_{11}a_{32}&a_{11}a_{22}-a_{12}a_{21}
\end{pmatrix}
\end{equation}
\inHfile{MAT\_INV33(invS, S)}{pprz\_simple\_matrix}