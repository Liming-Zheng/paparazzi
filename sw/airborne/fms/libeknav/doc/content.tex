\section{ENU to NED transformations}
I had the problem very often that I have to transform form ENU no NED. The simple conversion: ``Flip x and y and negate z'' doesn't work for quaternions or if you want to use matrix algebra.

\subsection{Matrix}
Flipping x and y and negating z is easy to express as a matrix:
\begin{equation}
R_{ENU2NED} = \begin{pmatrix}
0 & 1 &  0 \\
1 & 0 &  0 \\
0 & 0 & -1
\end{pmatrix}
\end{equation}
This works in both directions, since $ R_{ENU2NED} = \transp R_{ENU2NED} $.

\subsection{Quaternion}
It's easy to compute a quaternion out of the above rotation matrix.
\begin{equation}
\quat{ENU2NED} = \frac{1}{\sqrt 2} \begin{pmatrix}
0 \\ 1 \\ 1 \\ 0
\end{pmatrix} = \frac{1}{\sqrt 2} (i + j)
\end{equation}
This makes sense, since the real value = 0 represents a rotation about $180^{\circ}$ and the three values for the axis $ \vect v = \transp{\begin{pmatrix}1&1&0\end{pmatrix}}$ represent the axis of rotation.

\subsubsection*{Transforming a quaternion between ENU/NED}
If you want to cahnge a quaternion from NED to ENU or vice versa. It's not totally simple like for vectors. \\
If your quaternion consist of the values:
\begin{equation}
\quat{ECEF2NED} = \begin{pmatrix}a \\ b \\ c\\d\end{pmatrix}
\end{equation}
Then a transformation to ENU (NED) is made as following:
\begin{equation}
\quat{ECEF2NED} = \frac{1}{\sqrt 2} \begin{pmatrix}-b-c \\ a+d \\ a-d\\-b+c\end{pmatrix}
\end{equation}

Pay attention doing it twice! The multiplication of an NED to ENU quaternion with itself leads to
\begin{equation}
\quat{ECEF2NED} \quatprod \quat{ECEF2NED} \quatprod \begin{pmatrix}a \\ b \\ c\\d\end{pmatrix}= \begin{pmatrix}-a \\ -b \\ -c\\-d\end{pmatrix} .
\end{equation}
This is logically the same rotation, but mathemaically a different quaternion. So don't be confused if all values are negative :-)

\section{Initialisation}

\subsection{What about the standard deviation?}
\begin{figure}[h!]\begin{center}
	\begin{tikzpicture}[->,>=stealth',shorten >=1pt,auto,node distance=2cm,
                    semithick]
  \tikzstyle{every state}=[draw=black,text=white]

  \node		   (A)              {Noise};
  \node        (B) [below of=A] {Measurement};
  \node        (C) [below of=B] {attitude profile matrix};
  \node        (D) [below of=C] {``K'' - matrix};
  \node        (E) [below of=D] {Eigenvector};
  \node        (F) [below of=E] {Euler angle};

  \path (A) edge              node {Gaussian noise} (B)
  		(B) edge              node {weight of a single measurement} (C) 
  		(C) edge              node {(see the section below)} (D)
  		(D) edge              node {computational error} (E)
  		(E) edge              node {computational error} (F);

	\end{tikzpicture}
	\caption{Propagation of uncertainty}
	\label{Propagation of uncertainty}
\end{center}\end{figure}
First of all, every sensor (accelerometers $\vect a$ and magnetometers $\vect m$) has gaussian noise, that can be expressed as an additive error:
\begin{equation}
\vect a + \vect{\sigma_a} \quad \quad \vect m + \vect{\sigma_m}
\end{equation}
It can be asssumed that the error follows a standard deviation (has zero mean and is time-invariant).
The attitude profile matrix $ \mat B $ is the sum of the measurements with specific weights.
\begin{equation}\label{attitude profile matrix}
\mat B = \sum_{k=1}^n w_k \cdot \vect{W}_k \cdot \transp{\vect{V}_k} = w_a \sum_{k=1}^{n_a} \frac{\vect a_k}{\norm{a_k}} \cdot \transp{\vect{g}}  +  w_m \sum_{k=1}^{n_m} \frac{\vect m_k}{\norm{m_k}} \cdot \transp{\vect{h}}
\end{equation}
$n$ is the number of measurements, $w_k$ is the specific weight of a measurement, $\vect{W}_k$ the measured vector and $\vect{V}_k$ the reference direction, which belongs to the measured direction. Therefore $n_a$ is the number of acceleration measurements, $w_a$ is the (constant) weight of the acceleration measurements, $\vect{a}_k$ is a single acceleration observation and $\vect{g}$ is the gravity. $\vect{a}_k$ becomes normed. Similar for the magnetometer weight $w_m$, measurement $\vect {m}_k$, the magnetic field $\vect h$ and the amount of magnetometer measurements $n_m$. See the next section how the weight should be choosen.

The resulting error is 
\begin{equation}
\mat{\sigma_B} = \frac{n_a}{f_a} \frac{1}{\norm g_2} \vect{\sigma_a}\transp{\vect{g}} + \frac{n_m}{f_m} \frac{1}{\norm h_2} \vect{\sigma_m}\transp{\vect{m}}
\end{equation}

The error for the ``K''-matrix is easy to get by inserting $ \mat B + \mat {\sigma_B} $ into
\begin{equation}
\mat K = \begin{bmatrix}
trace(\mat B) & \transp{\vect Z} \\
\vect Z & \mat B + \transp{\mat B} - trace(\mat B) \mat I
\end{bmatrix}
\end{equation}











\subsection{choosing the best weight for the attitude profile matrix}
If you replace the single measurements in equation (\ref{attitude profile matrix}) with the real (and normed) measurements
\begin{equation}
\frac{\vect a_k + \vect{\sigma_a}}{\norm{a_k}}_2 \quad \quad \frac{\vect m_k + \vect{\sigma_m}}{\norm{m_k}}_2
\end{equation}
and assume that $\mat B$ has an error $\mat B +\mat{\sigma_B} $, you will get
\begin{equation}
\mat B +\mat{\sigma_B} = w_a \sum_{k=1}^{n_a} \frac{\vect a_k + \vect{\sigma_a}}{\norm{a_k}_2} \cdot \transp{\vect{g}}  +  w_m \sum_{k=1}^{n_m} \frac{\vect m_k + \vect{\sigma_m}}{\norm{m_k}_2} \cdot \transp{\vect{h}}
\end{equation}
\begin{equation}
\mat B +\mat{\sigma_B} = w_a \sum_{k=1}^{n_a} \frac{\vect a_k}{\norm{a_k}}_2 \cdot \transp{\vect{g}} + \frac{\vect{\sigma_a}}{\norm{a_k}}_2 \cdot \transp{\vect{g}} + w_m \sum_{k=1}^{n_m} \frac{\vect m_k}{\norm{m_k}}_2 \cdot \transp{\vect{h}} + \frac{\vect{\sigma_m}}{\norm{m_k}}_2 \cdot \transp{\vect{h}}
\end{equation}
\begin{equation}
\mat B +\mat{\sigma_B} = \underbrace{w_a \sum_{k=1}^{n_a} \frac{\vect a_k}{\norm{a_k}}_2\cdot \transp{\vect{g}} + w_m \sum_{k=1}^{n_m} \frac{\vect m_k}{\norm{m_k}}_2 \cdot \transp{\vect{h}}}_{\mat B} + w_a \sum_{k=1}^{n_a} \frac{\vect{\sigma_a}}{\norm{a_k}}_2 \cdot \transp{\vect{g}} + w_m \sum_{k=1}^{n_m} \frac{\vect{\sigma_m}}{\norm{m_k}}_2 \cdot \transp{\vect{h}}
\end{equation}
\begin{equation}
\mat{\sigma_B} = w_a \sum_{k=1}^{n_a} \frac{\vect{\sigma_a}}{\norm{a_k}}_2 \cdot \transp{\vect{g}} + w_m \sum_{k=1}^{n_m} \frac{\vect{\sigma_m}}{\norm{m_k}}_2 \cdot \transp{\vect{h}}
\end{equation}
$\norm{a_k}_2$ and $\norm{m_k}_2$ shouldn't vary that much and can be assumed as constant ($\norm{a}_2$ and $\norm{m}_2$). The equation reduces to:
\begin{equation}
\mat{\sigma_B} = w_a n_a \frac{\vect{\sigma_a}}{\norm{a}_2}\cdot \transp{\vect{g}} + w_m n_m \frac{\vect{\sigma_m}}{\norm{m}_2} \cdot \transp{\vect{h}}
\end{equation}

It would be nice, if it's possible to reduce this to a single value. To do that, we need a matrix norm. In this case, I choosed the Frobenius Norm:
\begin{align}
\norm{\mat{\sigma_B}}_{F} &= \norm{w_a n_a \frac{\vect{\sigma_a}}{\norm{a}_2}\cdot \transp{\vect{g}} + w_m n_m \frac{\vect{\sigma_m}}{\norm{m}_2} \cdot \transp{\vect{h}}}_{F}		\\
&\le \norm{w_a n_a \frac{\vect{\sigma_a}}{\norm{a}_2}\cdot \transp{\vect{g}}}_{F} + \norm{w_m n_m \frac{\vect{\sigma_m}}{\norm{m}_2} \cdot \transp{\vect{h}}}_{F}		\\
&= w_a n_a \frac{1}{\norm{a}_2}\cdot \norm{\vect{\sigma_a} \transp{\vect{g}}}_{F} + w_m n_m \frac{1}{\norm{m}_2} \cdot \norm{\vect{\sigma_m} \transp{\vect{h}}}_{F}
\end{align}

It is straight-forward to proove that $ \norm{\vect a \transp{\vect b}}_F = \norm a_2 \cdot \norm b_2 $
\begin{equation}
\norm{\mat{\sigma_B}}_{F} \le w_a n_a \frac{\norm g_2}{\norm a_2}\cdot \norm{\sigma_a}_2 + w_m n_m \frac{\norm h_2}{\norm m_2} \cdot \norm{\sigma_m}_2
\end{equation}

As you can see, the uncertainty depends on the following parameters:
\begin{itemize}
\item The weight of a measurement $w_a$ and $w_m$.
\item The number of measurements $n_a$ and $n_m$.
\item Something that I call a "measurement gain", $\frac{\norm{a}_2}{\norm{g}_2}$ and $\frac{\norm{m}_2}{\norm{h}_2}$, since it's the ratio between the true value and the measured value.
\item The maximum of the error $\sigma_a$ and $\sigma_m$.
\end{itemize}

This is not what I want. I don't want the error grow with the number of measurements or with the gain, that is related to the measruement device. If I choose
\begin{equation}
w_a = \frac{\norm a_2}{n_a \cdot \norm g_2} \quad and \quad w_m = \frac{\norm m_2}{n_m \cdot \norm h_2}
\end{equation}
I get something like
\begin{equation}
\norm{\mat{\sigma_B}}_{F} \le \norm{\sigma_a}_2 + \norm{\sigma_m}_2 \quad , 
\end{equation}
which looks much better. For the Frobenius norm of the attitude profile matrix the choosen weight leads to
\begin{equation}
\norm{\mat B}_{F} \le \frac{1}{n_a} \sum_{k=1}^{n_a} \norm{a_k}_2 + \frac{1}{n_m} \sum_{k=1}^{n_m} \norm{m_k}_2	\quad .
\end{equation}
That is an acceptable fact, since it helps to keep the matrix bound.
But because I want to do live-update of the attitude profile matrix I don't know the real amount of measurements $n_a$ and $n_m$. But I know the measurement frequencies $f_a$ and $f_m$, which are directly linked to them ($f = \tfrac n T $). So my final decision for the measurement weight is
\begin{equation}
w_a = \frac{\norm a_2}{f_a \cdot \norm g_2} \quad and \quad w_m = \frac{\norm m_2}{f_m \cdot \norm h_2} \quad .
\end{equation}
The resulting error is then
\begin{equation}
\mat{\sigma_B} = \frac{n_a}{f_a} \frac{1}{\norm g_2} \vect{\sigma_a}\transp{\vect{g}} + \frac{n_m}{f_m} \frac{1}{\norm h_2} \vect{\sigma_m}\transp{\vect{m}}
\end{equation}
or
\begin{equation}
\norm{\mat{\sigma_B}}_{F} \le \frac{n_a}{f_a} \norm{\sigma_a}_2 + \frac{n_m}{f_m} \norm{\sigma_m}_2 \quad , 
\end{equation}